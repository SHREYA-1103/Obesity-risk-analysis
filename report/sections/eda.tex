\chapter{Exploratory Data Analysis}\label{ch:exploratory-data-analysis}

This section presents a detailed exploratory data analysis of the dataset.
We examine data completeness, outlier presence, feature distributions, and sensitivity to extreme values.
Each subsection summarizes visual analyses and their corresponding insights.

%----------------------------------------------------------------


\section{Missing Value Detection}\label{sec:missing-value-detection}
\importTableFigure{"tables/data_eda_missing_counts.csv"}{Missing Counts}{eda_missing_counts}
\textbf{Observation:} There is not a single missing value in any of the columns.\\
\textbf{Inference:} The dataset is well-structured and complete.
No imputation, removal, or flagging of missing values is required.

%----------------------------------------------------------------


\section{Outlier Detection}\label{sec:outlier-detection}
\importTableFigure{"tables/data_eda_outlier_counts.csv"}{Outlier Counts}{eda_outlier_counts}
\textbf{Observation:} \textit{Age} and \textit{NCP} exhibit a significant number of values outside the interquartile range (IQR).\\
\textbf{Inference:} The presence of a large proportion of outliers suggests potential issues in data distribution.
However, dropping or imputing such a large portion of data would not be recommended.

%----------------------------------------------------------------


\section{Feature Distribution (Numerical)}\label{sec:feature-distribution-numerical}
\importPlotFigure{figures/plot_Feature Distribution (Numerical).png}{Numerical Features Distribution}{feature_distribution_numerical}

\textbf{Observation:}
\begin{itemize}
    \item \textit{Age} is moderately right-skewed, with most samples between 18–30 years.
    \item \textit{Height} and \textit{Weight} follow near-unimodal distributions; \textit{Weight} shows slight multimodality.
    \item Remaining features (\textit{FCVC}, \textit{NCP}, \textit{CH20}, \textit{FAF}, \textit{TUE}) exhibit distinct multimodality.
    \item \textit{FAF} (Physical Activity) is heavily right-skewed, dominated by low activity levels.
\end{itemize}

\textbf{Inference:}
The dataset combines continuous-like variables (\textit{Age, Height, Weight}) and discrete ordinal features (\textit{FCVC, NCP, CH20, TUE}).
Response clustering indicates limited variation within several behavior-related attributes.

%----------------------------------------------------------------


\section{Feature Distribution (Discrete)}\label{sec:feature-distribution-(discrete)}
\importPlotFigure{figures/plot_Feature Distribution (Discrete).png}{Discrete Features Distribution}{feature_distribution_discrete}

\textbf{Observation:}
\begin{itemize}
    \item \textit{Gender} is well-balanced.
    \item \textit{FHWO} and \textit{FAVC} show moderate skew towards “Yes”.
    \item \textit{SMOKE} and \textit{SCC} are highly skewed towards “No”.
    \item \textit{CAEC}, \textit{CALC}, and \textit{MTRANS} exhibit multi-category skewness.
\end{itemize}

\textbf{Inference:}
The population largely consists of individuals with a family history of overweight and high-calorie food habits, who rarely smoke or monitor calorie intake.
Public transport and occasional alcohol consumption are dominant lifestyle traits.

%----------------------------------------------------------------


\section{Feature Spread (Numerical)}\label{sec:feature-spread-(numerical)}
\importPlotFigure{figures/plot_Feature Spread (Numerical).png}{Numerical Feature Spread}{feature_spread_numerical}

\textbf{Observation:}
\begin{itemize}
    \item \textit{Age} shows concentrated distribution with long-tailed outliers up to 60s.
    \item \textit{Height} and \textit{Weight} display symmetric spread with minimal outliers.
    \item \textit{NCP} has a narrow IQR around 3 with widespread outliers.
    \item \textit{FCVC, CH20, FAF, TUE} have wide IQRs, suggesting diverse behaviors.
\end{itemize}

\textbf{Inference:}
Age and NCP exhibit concentrated cores surrounded by extreme values, while most other features show healthy spread.
This implies data variance primarily arises from behavioral rather than physical features.

%----------------------------------------------------------------


\section{Outlier Sensitivity (Numerical)}\label{sec:outlier-sensitivity-(numerical)}
\importPlotFigure{figures/plot_Outlier Sensitivity (Numerical).png}{Numerical Outlier Sensitivity}{outlier_sensitivity_numerical}

\textbf{Observation:}
\begin{itemize}
    \item \textit{Weight} and \textit{Age} show the highest deviation from the median.
    \item \textit{Height} has near-zero sensitivity.
    \item Other numerical features remain within a stable deviation range.
\end{itemize}

\textbf{Inference:}
\textit{Weight} and \textit{Age} are the most outlier-sensitive variables and will require focused treatment in the preprocessing phase.
\textit{Height} and other features remain largely robust.

%----------------------------------------------------------------


\section*{Summary of EDA Findings}
The dataset exhibits strong completeness and reasonable diversity across both physical and behavioral variables.
While outliers are present in \textit{Age} and \textit{NCP}, most features show stable distributions.
The observations suggest that feature scaling and selective outlier handling will enhance model robustness without major data loss.

