%\chapter*{Conclusion}\label{ch:conclusion}

    {
    \let\clearpage\relax
    \let\cleardoublepage\relax


    \chapter{Conclusion}
}

This project successfully demonstrated the application of a structured machine learning pipeline for predicting cardiovascular disease (CVD) risk based on obesity-related parameters.
Through systematic data exploration, feature engineering, and model evaluation, the study established a comprehensive framework that balances interpretability, performance, and robustness.

The results highlight that ensemble-based methods, particularly XGBoost, deliver the best predictive performance, achieving an accuracy of 99\% on the held-out test dataset.
This improvement was driven by careful preprocessing steps, including Box-Cox transformation for normalization, SMOTE-based oversampling for class balance, and appropriate encoding strategies for categorical variables.
The experiments also revealed that retaining certain outliers improved model generalization, reflecting the realistic variability of health-related data.

Comparative analyses confirmed that while linear models provided valuable interpretability, they lacked the flexibility to capture the complex non-linear interactions among features such as age, dietary habits, and activity levels.
Tree-based and boosting algorithms, on the other hand, effectively modeled these relationships and maintained robustness against noise.

In conclusion, this work demonstrates that thoughtfully engineered preprocessing combined with advanced ensemble learning can yield highly reliable health risk prediction models.
The findings underscore the potential of machine learning in preventive healthcare analytics and provide a foundation for future extensions, such as integrating explainability frameworks (e.g., SHAP or LIME) and testing the model on real-world clinical datasets to enhance its applicability and trustworthiness