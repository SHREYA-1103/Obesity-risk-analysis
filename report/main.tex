%! Author = Anirudh Sharma & Shreya Gupta
%! Date = 18-10-2025
%! Mail Addresses = mcxiv14@gmail.com, shreya.gupta.0624@gmail.com

\documentclass[12pt,a4paper]{report}

% =======================
% FONTS
% =======================
\usepackage{fontspec}       
\setmainfont{TeX Gyre Termes}
\usepackage[colorlinks=true, urlcolor=blue, linkcolor=black]{hyperref}

% =======================
% MODULAR SETTINGS
% =======================
\usepackage[utf8]{inputenc}
\usepackage[T1]{fontenc}
\usepackage{lmodern}
\usepackage[english]{babel}
\usepackage{graphicx}
\usepackage{amsmath, amssymb, amsfonts}
\usepackage{booktabs}
\usepackage{hyperref}
\usepackage{fancyhdr}
\usepackage{geometry}
\usepackage{xcolor}
\usepackage{titlesec}
\usepackage{setspace}

% Optional typography enhancements
\usepackage{microtype}

% Line spacing
\onehalfspacing

\geometry{
    a4paper,
    left=25mm,
    right=20mm,
    top=25mm,
    bottom=25mm,
    headheight=15pt
}

% Section title styling
\titleformat{\chapter}[hang]{\bfseries\Large}{\thechapter.}{1em}{}
\titleformat{\section}[hang]{\bfseries\large}{\thesection}{0.75em}{}
\titleformat{\subsection}[hang]{\bfseries\normalsize}{\thesubsection}{0.5em}{}

\pagestyle{fancy}
\fancyhf{} % clear all header and footer fields

% Define headers
\fancyhead[L]{\nouppercase{\leftmark}}
\fancyhead[R]{\thepage}

% Optional footer line
\fancyfoot[C]{\textcolor{gray}{ML Report – \today}}

% Header/footline separator
\renewcommand{\headrulewidth}{0.4pt}
\renewcommand{\footrulewidth}{0.2pt}

\newcommand{\HRule}{\rule{\linewidth}{0.5mm}}

\newcommand{\keyword}[1]{\textbf{\textit{#1}}}
\newcommand{\todo}[1]{\textcolor{red}{[TODO: #1]}}

% Example shorthand for commonly used symbols
\newcommand{\R}{\mathbb{R}}
\newcommand{\E}{\mathbb{E}}
\newcommand{\Var}{\mathrm{Var}}

\newcommand{\importTableFigure}[3]{
    \begin{figure}[ht]
    \renewcommand{\arraystretch}{0.9}
    \setlength{\tabcolsep}{3pt}
    \centering
    \footnotesize
    \csvautobooktabular[
    respect all,
%    late after line=\\,
    table head=\toprule\csvlinetotablerow\\\midrule,
    table foot=\\\bottomrule,
    before line=\bfseries
    ]{#1}
    \caption{#2}
    \label{table:#3}
    \end{figure}
}

\newcommand{\importPlotFigure}[3]{%
        {%
        \begin{figure}[ht]
        \centering
        \includegraphics[width=0.5\textwidth]{#1}
        \caption{#2}
        \label{plot:#3}
        \end{figure}
    }%
}



% =======================
% BIBLIOGRAPHY
% =======================
\usepackage[backend=biber,style=ieee]{biblatex} % IEEE style
\addbibresource{references.bib}

% =======================
% DOCUMENT
% =======================
\begin{document}

    % -----------------------
    % Title Page
    % -----------------------
    {
        \let\clearpage\relax
        \begin{titlepage}
    \centering

    \HRule \\[0.5cm]
    {\Huge \bfseries Project Report \\[0.4cm]
    \Large A Comprehensive Study on Machine Learning Algorithms}\\[0.5cm]
    \HRule \\[1.5cm]

    % Authors section
    \begin{minipage}{0.45\textwidth}
        \centering
        \textbf{Anirudh Sharma} \\[0.2cm]
    \end{minipage}
    \hfill
    \begin{minipage}{0.45\textwidth}
        \centering
        \textbf{Shreya Gupta} \\[0.2cm]
    \end{minipage}

    \vfill

    \textbf{Institution:} International Institute of Information Technology \\[0.5cm]
    \textbf{Date:} \today \\[2cm]

\end{titlepage}

    }

    % -----------------------
    % Abstract
    % -----------------------
    {
        \let\clearpage\relax
        \chapter*{Abstract}
        \addcontentsline{toc}{chapter}{Abstract}
        % \chapter{Abstract}

% TODO: Do we even want to keep it?
%  If yes, better move it to title page
        This project focuses on developing a machine-learning-based predictive model to assess the risk of cardiovascular diseases (CVD) in individuals based on obesity-related factors.
        The study uses an Obesity-CVD Risk dataset containing both numerical and categorical health indicators.
        An extensive Exploratory Data Analysis (EDA) was conducted to understand the data distribution, detect and handle missing values and outliers, and identify relationships between features and the target variable.
        Box-Cox transformation was applied to normalize skewed numerical attributes such as Age while categorical variables were encoded appropriately-One-Hot Encoding for logistic regression and Label Encoding for tree-based models.
        Oversampling using SMOTE was performed to address class imbalance.
        Feature engineering and transformation were followed by comparative model development using various algorithms, including Logistic Regression, Decision Tree, Random Forest, AdaBoost, K-Nearest Neighbors, Na\"ive Bayes, XGBoost, and LightGBM. Grid Search Cross-Validation was applied for hyperparameter optimization of most models, while Optuna was used for advanced tuning of XGBoost and LightGBM with Stratified K-Fold validation.
        Model performance was evaluated on a separate test dataset, where XGBoost achieved the highest accuracy of ~91.1\% on the validation dataset and ~93.2\% on the test dataset on Kaggle, demonstrating its strong predictive capability and robustness for health risk classification tasks.
        This study highlights the significance of feature transformation and advanced hyperparameter tuning in improving model performance for medical risk prediction, offering a scalable framework for preventive healthcare analytics.
        The complete codebase for this project on GitHub can be accessed via:
        \href{https://github.com/SHREYA-1103/Obesity-risk-analysis}{\underline{\textcolor{blue}{https://github.com/SHREYA-1103/Obesity-risk-analysis}}}

    }

    % -----------------------
    % Table of Contents
    % -----------------------
    \tableofcontents
    \newpage

    % -----------------------
    % Chapters
    % -----------------------
    \chapter{Introduction}\label{ch:introduction}


Cardiovascular diseases (CVDs) continue to be the leading cause of death across the world, accounting for nearly one-third of all global mortalities each year.
They represent a group of disorders of the heart and blood vessels that often arise due to a combination of genetic, metabolic, and lifestyle-related factors.
Among these, obesity has emerged as one of the most significant risk factors.
Obesity is a complex medical condition characterized by excessive accumulation of body fat, typically resulting from an imbalance between calorie intake and energy expenditure.
It contributes to various metabolic disturbances such as high blood pressure, elevated cholesterol levels, insulin resistance, and inflammation-all of which are strongly associated with cardiovascular complications.
The growing prevalence of obesity across all age groups and regions has made it a critical public health challenge.
Consequently, the early prediction and prevention of CVDs among obese individuals have become an important area of research in both medicine and data science.

Traditional approaches to cardiovascular risk assessment, such as the Framingham Risk Score or BMI-based methods, rely on fixed statistical formulas or thresholds derived from population studies.
While these models are useful, they often oversimplify complex biological interactions and fail to account for nonlinear relationships between multiple risk factors.
In contrast, \textbf{modern machine learning (ML)} methods have demonstrated significant potential in analyzing large, multidimensional datasets to uncover intricate patterns that might not be visible through classical techniques.
By learning from data, ML models can automatically adapt to complex relationships between variables and generate accurate, data-driven predictions.
This makes them especially valuable in healthcare, where early diagnosis and risk prediction can directly improve patient outcomes and reduce healthcare burdens.

The application of machine learning in medical diagnosis and preventive care has expanded rapidly in recent years.
ML models have been successfully employed in various domains such as cancer detection, diabetes prediction, neurological disorder identification, and cardiovascular risk assessment.
These models leverage data collected from clinical measurements, imaging, laboratory results, and lifestyle attributes to identify patterns indicative of disease presence or progression.
For conditions like CVD, which develop gradually and involve multiple interacting risk factors, machine learning can serve as an intelligent support system that helps in early detection and timely intervention.
It allows for individualized risk stratification, where personalized recommendations can be made based on the patient’s unique health profile.
The motivation for this project arises from the need to create a reliable, data-driven prediction model that can identify cardiovascular disease risk using obesity-related parameters.
With an increase in sedentary lifestyles, unhealthy eating habits, and stress levels, obesity rates have surged globally, making cardiovascular diseases a frequent consequence.
However, due to the complex nature of these conditions, predicting who is at risk remains a challenge.
Therefore, this project aims to develop and evaluate several machine learning models that can predict CVD risk accurately by learning from obesity-linked behavioral and physiological factors.
Such predictive systems can act as powerful tools for public health agencies, fitness platforms, and healthcare professionals by enabling early risk identification and facilitating preventive care strategies.

The dataset used in this study comprises a combination of \textbf{demographic, behavioral, and physical health attributes} associated with obesity and cardiovascular conditions.
It includes features such as age, gender, dietary habits, water intake, frequency of vegetable consumption, physical activity level, and other health indicators.
Since the dataset contains both numerical and categorical data, it provides an excellent opportunity to explore comprehensive preprocessing and transformation strategies.
The first step involved conducting an \textbf{Exploratory Data Analysis (EDA)} to understand the underlying structure and characteristics of the data.
This included examining dataset dimensions, column types, summary statistics, and distribution patterns.
Handling missing values and detecting outliers were essential parts of this phase to ensure that the data quality was suitable for model training.
Visualizations such as box plots and histograms were employed to observe outlier behavior and assess normality in distributions.
After understanding the data, transformations were performed to improve model compatibility and performance.
Skewed numerical variable like Age was subjected to \textbf{Box–Cox transformation}, which helps stabilize variance and normalize distributions.
This is particularly beneficial for algorithms that assume Gaussian distributions, such as logistic regression.
For categorical features, encoding techniques were applied depending on the model type.
\textbf{One-Hot Encoding (OHE)} was used for logistic regression to prevent imposing any ordinal relationship among categories, whereas \textbf{Label Encoding} was applied for tree-based models such as Random Forest, XGBoost, and LightGBM, which can handle categorical features directly and are not sensitive to arbitrary label values.
Since imbalanced datasets are common in medical predictions-where one class (e.g., healthy individuals) often dominates over another (e.g., individuals with disease)- \textbf{Synthetic Minority Oversampling Technique (SMOTE)} was employed to balance the class distribution.
This approach synthetically generates new samples for the minority class based on feature-space similarities, ensuring that all models receive balanced training data and reducing bias toward the majority class.
Data scaling was also performed using \textbf{Standard Scaler} to normalize numerical feature ranges, preventing features with large magnitudes from dominating others during model training.

The core objective of this project was to design a robust pipeline for model training and evaluation using multiple machine learning algorithms.
Various models were tested, including \textbf{Logistic Regression, Decision Tree, Random Forest, AdaBoost, K-Nearest Neighbors (KNN), Naïve Bayes, XGBoost, and LightGBM}. To achieve fair and optimal comparisons, hyperparameter tuning was carried out for each model.
Grid Search Cross-Validation was applied for models such as Logistic Regression, Decision Tree, Random Forest, AdaBoost, KNN, and Naïve Bayes to systematically explore parameter combinations and identify the configuration that yielded the highest validation accuracy.
For advanced ensemble models such as \textbf{XGBoost and LightGBM, the Optuna} framework was utilized.
Optuna provides an efficient and automated optimization process that uses Bayesian sampling and pruning techniques, allowing faster convergence toward the best-performing hyperparameters.
Stratified K-Fold Cross-Validation was also implemented to ensure that each fold maintained the same class distribution as the original dataset, leading to reliable and unbiased performance estimates.

The prepared data was divided into training, validation, and testing subsets.
Each model was trained on the training data and validated using cross-validation to monitor consistency.
After hyperparameter optimization, the models were tested on unseen data to evaluate generalization.
The final step involved making predictions on a separate test dataset for Kaggle submission, where accuracy was calculated to measure real-world applicability.
Among all models tested, \textbf{XGBoost achieved the highest accuracy of ~91.1\%} on the validation dataset, and \textbf{~93.27\%} on the test dataset on Kaggle, outperforming other algorithms in terms of precision and stability.
The strong performance of XGBoost can be attributed to its gradient-boosting framework, regularization techniques, and ability to capture non-linear feature interactions effectively.
The results of this study highlight the importance of rigorous preprocessing, feature transformation, and hyperparameter tuning in improving predictive performance.
Each step in the pipeline-from data cleaning and encoding to model optimization- played a critical role in achieving high accuracy.
Moreover, the comparison between models revealed that ensemble techniques like XGBoost and LightGBM outperform simpler models because they can learn from multiple weak learners to produce a strong final prediction.
These findings align with existing research trends in healthcare analytics, where gradient boosting models are widely used for medical risk assessment due to their robustness and interpretability.

Beyond model performance, the significance of this project lies in its potential \textbf{realworld applications}.
Accurate risk prediction systems can serve as early-warning tools that empower individuals and healthcare providers to take preventive measures before severe complications arise.
Integrating such models into healthcare platforms or wearable health-tracking applications can promote personalized medicine, continuous monitoring, and data-driven interventions.
Furthermore, this project demonstrates a generalizable framework for handling complex health datasets- combining EDA, data transformation, feature engineering, class balancing, and model optimization-that can be extended to other domains such as diabetes detection, nutrition assessment, and metabolic disorder prediction.

    \chapter{Methodology}\label{ch:methodology}


In general, a standard methodology for machine learning projects follows a linear process - loading the dataset, visualizing it, performing data cleaning and preprocessing, training models, and finally generating the submission outputs.

However, building upon insights from prior works on similar data-pipelining tasks, this project adopts a more mature and iterative framework rather than a purely linear one.
For instance, data visualization is no longer treated as a standalone phase; instead, it is an ongoing process integrated throughout the workflow - from data loading and preprocessing to model training and evaluation.
This continuous visualization strategy enables early detection of anomalies and fosters a deeper understanding of model–data interactions at every stage.

Furthermore, the need for abstraction and modularity becomes evident when conducting comparative studies involving multiple models and experiments.
Several experiments share similar configurations or preprocessing routines, and to handle this efficiently, the project employs helper methods and modular utilities that encapsulate repeated processes.
This abstraction ensures code clarity, minimizes redundancy, and simplifies systematic experimentation.

In this report, a custom pipelining infrastructure is presented to enhance transparency and control over each step of the process.
Unlike the conventional scikit learn pipeline, this system provides richer insights into the effectiveness of individual preprocessing strategies, offering fine-grained visualization of their impact on model performance.

Lastly, leveraging lessons from earlier projects, the implementation maintains rigorous control over reproducibility aspects such as random seed management, model versioning, and grid search configurations, ensuring consistent and reliable experimentation across runs.

\importPlotFigure{figures/plot_Model Comparison.png}{Model Comparison}{model_time_comparison}


\section{Dataset Description}\label{sec:dataset-description}
The data loading phase follows a straightforward and minimal approach.
The primary dataset is loaded into {ds_source}, and the submission dataset into {ds_test}.
An additional preprocessing step involves renaming the column family_history_with_overweight to FHWO, aligning it with the naming convention of other acronym-based attributes.
This minor modification simplifies table formatting, improves readability, and ensures consistency across visualizations and statistical summaries.
The primary dataset is loaded into \texttt{\{ds\_source\}} and the submission dataset into \texttt{\{ds\_test\}}.


\section{Exploratory Data Analysis (EDA)}\label{sec:exploratory-data-analysis}
The Exploratory Data Analysis (EDA) phase focuses on identifying missing values, outliers, and other statistical anomalies within the dataset.
Although this section might appear procedural, its purpose extends beyond aesthetic visualization - it aims to extract meaningful insights that guide data preprocessing and modeling.

A variety of plots and statistical summaries are generated to ensure a comprehensive understanding of data distributions, correlations, and feature relationships.
Care is taken to emphasize interpretability - each visualization contributes directly to understanding the dataset’s behavior and influences the decisions made during feature engineering and preprocessing.


\section{Feature Engineering & Preprocessing}\label{sec:data-preprocessing}
The feature engineering and preprocessing stage forms the backbone of the entire modeling workflow, translating raw data into structured, meaningful representations suitable for machine learning algorithms.
All transformations applied at this stage are informed by insights drawn from the exploratory data analysis (EDA). Rather than mutating the original data source, the transformations are encapsulated within modular units known as Pipeline Operations (POPs), which maintain full traceability and reversibility of each preprocessing step.

\subsection{Transformation Strategy}\label{subsec:transformation-strategy}
The primary objective of this phase is to enhance the interpretability and predictive strength of the dataset while ensuring model compatibility across diverse algorithmic families.
Both numerical and categorical attributes undergo distinct transformation and encoding schemes, tailored to the type of model being trained (linear or treebased).

\textbf{Numerical Transformations}

During distribution analysis, several continuous variables such as Age, NCP, CH₂O, and FCVC displayed moderate skewness and non-Gaussian patterns.
To address this, multiple transformation techniques - logarithmic, square root, and Box–Cox transformations - were systematically tested.

For the Age attribute, the Box–Cox transformation yielded the most statistically balanced distribution, effectively reducing skewness and improving normality, whereas the log and square root transformations provided minimal improvement.

For Weight, none of the tested transformations led to a significant improvement in distribution symmetry; thus, the original scale was retained.

These transformations were validated through post-transformation histograms, Q–Q plots, and normality tests to ensure stability before being incorporated into the final pipeline.

\textbf{Categorical Transformations}

The dataset contained multiple categorical features, each influencing the model differently depending on the algorithm used.
To accommodate this, dual-encoding strategies were applied:

For linear models such as Logistic Regression, categorical variables were One-Hot Encoded (OHE) to maintain orthogonality and prevent unintended ordinal relationships.
Additionally, a specialized soft one-hot encoding (soft OHE) technique was applied to features like CH₂O, which exhibited discrete peaks within a continuous range.

For tree-based models such as Decision Tree, Random Forest, XGBoost, and LightGBM, categorical features were Label Encoded, preserving computational efficiency and minimizing feature dimensionality.

This dual-encoding design ensured that each model family received the most suitable representation of categorical information, maximizing predictive performance without introducing redundancy.

\textbf{Feature Derivation}

Beyond transformation, several derived features were engineered to capture latent relationships among existing attributes.
These were designed based on domain intuition about obesity and cardiovascular risk factors.
Derived features included:

BMI – Calculated from height and weight to provide a normalized indicator of body composition.

Water\_Intake\_per\_Meal – Derived from CH₂O and NCP to approximate hydration behavior per meal.

Activity\_to\_Tech\_Ratio – Ratio of physical activity levels to technology usage time, representing lifestyle balance.

Healthy\_Lifestyle\_Score – Weighted aggregate of positive lifestyle indicators such as high FCVC, adequate CH₂O, and regular physical activity.

Has\_FamilyRisk\_and\_FAVC – Interaction feature capturing the combined impact of genetic predisposition and high-calorie food consumption.

Calorie\_Monitoring\_Interaction – Interaction between food frequency (FAF) and calorie monitoring (CAEC), approximating dietary awareness.

These engineered features significantly improved the feature space’s expressiveness and allowed models to capture nuanced behavioral and physiological interactions relevant to obesity and cardiovascular risk.

\textbf{Scaling and Standardization}

To ensure uniform feature contribution, StandardScaler was applied to all numerical features.
This transformation standardized variables to zero mean and unit variance, a critical step for models sensitive to feature magnitudes such as Logistic Regression and K-Nearest Neighbors.
For tree-based models, which are scale-invariant, raw or label-encoded data were retained to preserve interpretability.

\textbf{Oversampling and Data Balance}

Given the class imbalance observed in obesity–CVD categories, Synthetic Minority Oversampling Technique (SMOTE) was used to generate synthetic minority samples\. The sample size increased from approximately 15,000 to 20,000.
This significantly enhanced the model’s ability to recognize minority patterns, resulting in up to 3\% accuracy improvements in XGBoost, the best-performing model.

All preprocessing operations were implemented as POPs, enabling modular combination, visualization, and reproducibility across experiments.


\section{Setups and Helpers}\label{sec:setups-and-helpers}
Although this section does not directly appear in the final report output, it plays a foundational role within the notebook implementation.
Executed immediately after the import statements, it handles several preliminary configurations essential for smooth experimentation.
These include:

Initialization of the notification and logging system

PyPlot and visualization styling adjustments

Random seed management for reproducibility

Notebook display and formatting controls

Model training, saving, and submission automation

Definition and registration of POP operations

\subsection{Pipeline Operations (POPs)}\label{subsec:pipeline-operations-(pops)}
Each POP is implemented as a modular function that takes a data object (with optional configuration parameters) and returns a tuple containing the transformation configuration and the processed dataset.
This design encourages composability and experimentation across different transformation strategies.

The implemented POPs include:

\begin{itemize}
    \item pop\_drop\_column
    \item pop\_log\_transform
    \item pop\_root\_transform
    \item pop\_box\_cox\_transform
    \item pop\_one\_hot\_encode
    \item pop\_soft\_ohe
    \item pop\_binarize
    \item pop\_standardize
    \item pop\_minmax
    \item pop\_ordinal\_encode
    \item pop\_round
    \item pop\_derived\_features
\end{itemize}

This modular approach enables quick mixing and matching of transformations, facilitating the discovery of the most effective preprocessing combinations through systematic experimentation.

\subsection{Helper Utilities}\label{subsec:helper-utilities}
The POP infrastructure is supported by several helper functions designed to streamline experimentation and simplify configuration management.
These include:

\textbf{compose\_pop} – Combines multiple POPs into a single unified operation.

\textbf{prepare\_pop} – Finalizes a POP by fixing its configuration parameters.

\textbf{select\_pipeline\_variation} – Chooses the most promising pipeline configuration among various permutations.

\textbf{apply\_pipeline} – Applies a series of pipeline variations to the dataset and evaluates their effects.

Together, these functions ensure the infrastructure remains modular, scalable, and adaptive to multiple model setups.

\subsection{Specialized POPs}\label{subsec:specialized-pops}
Among the implemented transformations, two specialized POPs deserve particular mention:

\textbf{pop\_soft\_ohe} – Performs a “soft” one-hot encoding for features exhibiting discrete peaks within an otherwise continuous distribution.
For example, in the CH₂O feature, distinct spikes occur at values 1, 2, and 3.
This transformation isolates these regions, creating dedicated features that capture each peak more effectively.

\textbf{pop\_derived\_features} – Generates higher-level derived features from existing columns.
These engineered features capture latent relationships and health-related behavioral indicators such as:

BMI

Water\_Intake\_per\_Meal

Activity\_to\_Tech\_Ratio

Healthy\_Lifestyle\_Score

Has\_FamilyRisk\_and\_FAVC

Calorie\_Monitoring\_Interaction

Together, these components establish a robust experimental foundation, ensuring that every transformation and pipeline decision can be systematically analyzed, visualized, and reproduced across experiments.


\section{Model Training and Evaluation}\label{sec:model-training-and-evaluation}
The final phase of the methodology involves model construction, hyperparameter optimization, and performance evaluation.
The training framework was designed to maintain consistency across experiments while enabling model-specific customization.

\subsection{Data Splitting}\label{subsec:data-splitting}
The preprocessed dataset was split into training (70\%), and validation (30\%) subsets using Stratified Sampling to maintain proportional class representation across all splits.
Separate data configurations were maintained for:

Linear models (Logistic Regression, KNN, Naive Bayes): using OHE-transformed datasets.

Tree-based models (Decision Tree, Random Forest, AdaBoost, XGBoost, LightGBM): using label-encoded datasets.

\importPlotFigure{figures/plot_Feature Distribution (Discrete).png}{Discrete Features Distribution}{feature_distribution_discrete}

\subsection{Model Selection and Training}\label{subsec:model-selection-and-training}
A diverse suite of models was implemented to evaluate various learning paradigms:

Logistic Regression – Served as the baseline linear classifier, evaluated using regularization (L1, L2) and solver variations.

Decision Tree Classifier – Provided interpretability and served as the foundation for ensemble models.

Random Forest Classifier – Leveraged ensemble bagging to reduce variance and improve robustness.

AdaBoost Classifier – Introduced boosting-based iterative learning to handle hard-toclassify instances.

K-Nearest Neighbors (KNN) – Captured local instance-based decision boundaries.

Naive Bayes – Provided a probabilistic baseline assuming feature independence.

XGBoost Classifier – Gradient boosting model optimized for speed and accuracy, offering strong generalization capability.

LightGBM Classifier – Gradient boosting framework optimized for memory and computational efficiency.
\\ \\
All models were trained on the training set and evaluated on the validation and test sets.

\subsection{Hyperparameter Optimization}\label{subsec:hyperparameter-optimization}
Different optimization approaches were used depending on model complexity:

For Logistic Regression, Decision Tree, Random Forest, AdaBoost, KNN, and Naive Bayes, Grid Search Cross-Validation was used to explore parameter combinations systematically.

For XGBoost and LightGBM, Optuna was employed for Bayesian hyperparameter optimization using Stratified K-Fold Cross-Validation (k=5).
Optuna dynamically adjusted hyperparameters based on prior results, efficiently converging to optimal configurations.

This hybrid tuning strategy balanced exploration and computational efficiency while ensuring consistency across model evaluations.

\subsection{Evaluation Metrics}\label{subsec:evaluation-metrics}
Model performance was primarily evaluated using accuracy, given the balanced dataset post-SMOTE.
Additional metrics such as precision, recall, F1-score, and confusion matrix were used for detailed class-level analysis.

Each model’s predictions were also validated on a separate Kaggle submission dataset, ensuring the robustness of the trained models on unseen data.

The best accuracy of 93.2\% was achieved using XGBoost, followed closely by LightGBM, both of which benefited from the rich feature set, proper handling of categorical encodings, and SMOTE-based class balance.

    \chapter{Dataset Description}\label{ch:dataset-description}


The dataset used in this study, titled \textit{Obesity or CVD Risk (Classify)}, contains information from 15,533 individuals aged between 14 and 61 from Mexico, Peru, and Colombia.
The data were collected using an online survey assessing demographic information, eating habits, and physical condition.
The dataset includes 18 columns: 16 features, 1 target variable (\texttt{WeightCategory}), and 1 identifier column.

The features are grouped as follows:

\begin{enumerate}
    \item \textbf{Demographic variables:} Gender, Age, Height, Weight.
    \item \textbf{Eating habits:} Frequent consumption of high-caloric food (FAVC), Frequency of vegetable consumption (FCVC), Number of main meals (NCP), Food consumption between meals (CAEC), Water consumption (CH20), Alcohol consumption (CALC).
    \item \textbf{Physical condition:} Calories consumption monitoring (SCC), Physical activity frequency (FAF), Time using technology devices (TUE), Transportation method (MTRANS).
\end{enumerate}

The dataset contains both numerical and categorical data, enabling analysis using classification, regression, clustering, and association algorithms.

\importTableFigure{tables/data_dataset_statistics.csv}{Dataset Summary Statistics}{data_dataset_statistics}

\importTableFigure{tables/data_dataset_info.csv}{Dataset Information (Structure Overview)}{data_dataset_info}

\importTableFigure{tables/data_dataset_dtypes.csv}{Feature Data Types}{data_dataset_dtypes}

%\importTableFigure{tables/data_dataset_head.csv}{Sample Data Records}{data_dataset_head}

\importTableFigure{tables/data_dataset_describe.csv}{Statistical Summary of Features}{data_dataset_describe}

    \chapter{Exploratory Data Analysis}\label{ch:exploratory-data-analysis}

This section presents a detailed exploratory data analysis of the dataset.
We examine data completeness, outlier presence, feature distributions, and sensitivity to extreme values.
Each subsection summarizes visual analyses and their corresponding insights.

%----------------------------------------------------------------


\section{Missing Value Detection}\label{sec:missing-value-detection}
\importTableFigure{"tables/data_eda_missing_counts.csv"}{Missing Counts}{eda_missing_counts}
\textbf{Observation:} There is not a single missing value in any of the columns.\\
\textbf{Inference:} The dataset is well-structured and complete.
No imputation, removal, or flagging of missing values is required.

%----------------------------------------------------------------


\section{Outlier Detection}\label{sec:outlier-detection}
\importTableFigure{"tables/data_eda_outlier_counts.csv"}{Outlier Counts}{eda_outlier_counts}
\textbf{Observation:} \textit{Age} and \textit{NCP} exhibit a significant number of values outside the interquartile range (IQR).\\
\textbf{Inference:}
The presence of a large proportion of outliers suggests potential issues in the data distribution.
However, dropping or imputing such a large portion of data would not be recommended, as it could lead to the loss of valuable variance and distort the underlying population structure.
Instead of aggressively removing these data points, a more balanced approach was adopted to preserve data diversity while improving model generalization.

To address class imbalance and enhance the dataset’s representativeness, oversampling using the Synthetic Minority Oversampling Technique (SMOTE) was applied.
This technique increased the total number of samples from approximately 15,000 to 20,000, effectively improving the model’s ability to learn from minority classes.
Interestingly, even for the best-performing model— XGBoost—SMOTE contributed to nearly a 3\% improvement in accuracy, indicating that data balancing played a crucial role in enhancing classification performance.

Although experiments were conducted by removing extreme outliers and retraining the models, the results showed a decline in generalization capability.
The models trained on the cleaned dataset performed well on training data but failed to maintain consistent accuracy on validation and test sets.
This behavior suggested that the apparent outliers might, in fact, carry meaningful variations essential for capturing the inherent noise and diversity within the dataset.

Moreover, it is important to note that the Obesity–CVD Risk dataset used in this project was itself derived through oversampling and controlled noise addition to an original dataset of around 2,000 data points.
Therefore, a certain degree of randomness and non-uniformity in feature distributions was expected.
These artificial augmentations were introduced to expand the dataset and prevent overfitting during model training.
Consequently, preserving mild outlier behavior aligned with the dataset’s generative nature and ensured that the model learned to handle noisy, realworld data effectively.

%----------------------------------------------------------------


\section{Feature Distribution (Numerical)}\label{sec:feature-distribution-numerical}
\importPlotFigure{figures/plot_Feature Distribution (Numerical).png}{Numerical Features Distribution}{feature_distribution_numerical}

\textbf{Observation:}
\begin{itemize}
    \item \textit{Age} is moderately right-skewed, with most samples between 18–30 years.
    \item \textit{Height} and \textit{Weight} follow near-unimodal distributions; \textit{Weight} shows slight multimodality.
    \item Remaining features (\textit{FCVC}, \textit{NCP}, \textit{CH20}, \textit{FAF}, \textit{TUE}) exhibit distinct multimodality.
    \item \textit{FAF} (Physical Activity) is heavily right-skewed, dominated by low activity levels.
\end{itemize}

\textbf{Inference:}
The dataset combines continuous-like variables (\textit{Age, Height, Weight}) and discrete ordinal features (\textit{FCVC, NCP, CH20, TUE}).
Response clustering indicates limited variation within several behavior-related attributes.

%----------------------------------------------------------------


\section{Feature Distribution (Discrete)}\label{sec:feature-distribution-(discrete)}
\importPlotFigure{figures/plot_Feature Distribution (Discrete).png}{Discrete Features Distribution}{feature_distribution_discrete}

\textbf{Observation:}
\begin{itemize}
    \item \textit{Gender} is well-balanced.
    \item \textit{FHWO} and \textit{FAVC} show moderate skew towards “Yes”.
    \item \textit{SMOKE} and \textit{SCC} are highly skewed towards “No”.
    \item \textit{CAEC}, \textit{CALC}, and \textit{MTRANS} exhibit multi-category skewness.
\end{itemize}

\textbf{Inference:}
The population largely consists of individuals with a family history of overweight and high-calorie food habits, who rarely smoke or monitor calorie intake.
Public transport and occasional alcohol consumption are dominant lifestyle traits.

%----------------------------------------------------------------


\section{Feature Spread (Numerical)}\label{sec:feature-spread-(numerical)}
\importPlotFigure{figures/plot_Feature Spread (Numerical).png}{Numerical Feature Spread}{feature_spread_numerical}

\textbf{Observation:}
\begin{itemize}
    \item \textit{Age} shows concentrated distribution with long-tailed outliers up to 60s.
    \item \textit{Height} and \textit{Weight} display symmetric spread with minimal outliers.
    \item \textit{NCP} has a narrow IQR around 3 with widespread outliers.
    \item \textit{FCVC, CH20, FAF, TUE} have wide IQRs, suggesting diverse behaviors.
\end{itemize}

\textbf{Inference:}
Age and NCP exhibit concentrated cores surrounded by extreme values, while most other features show healthy spread.
This implies data variance primarily arises from behavioral rather than physical features.

%----------------------------------------------------------------


\section{Outlier Sensitivity (Numerical)}\label{sec:outlier-sensitivity-(numerical)}
\importPlotFigure{figures/plot_Outlier Sensitivity (Numerical).png}{Numerical Outlier Sensitivity}{outlier_sensitivity_numerical}

\textbf{Observation:}
\begin{itemize}
    \item \textit{Weight} and \textit{Age} show the highest deviation from the median.
    \item \textit{Height} has near-zero sensitivity.
    \item Other numerical features remain within a stable deviation range.
\end{itemize}

\textbf{Inference:}
\textit{Weight} and \textit{Age} are the most outlier-sensitive variables and will require focused treatment in the preprocessing phase.
\textit{Height} and other features remain largely robust.

%----------------------------------------------------------------


\section*{Summary of EDA Findings}
The dataset exhibits strong completeness and reasonable diversity across both physical and behavioral variables.
While outliers are present in \textit{Age} and \textit{NCP}, most features show stable distributions.
The observations suggest that feature scaling and selective outlier handling will enhance model robustness without major data loss.


    \chapter{Data Preprocessing & Feature Engineering}\label{ch:feature-engineering}


This is one of the most crucial stages in the machine learning pipeline, as it directly influences how effectively models can learn patterns and relationships from the data.
This process involves transforming raw features into formats that enhance model interpretability, reduce bias, and improve predictive performance.
In this project, several transformations were applied to handle skewed numerical variables, encode categorical variables, and ensure that both linear and tree-based models received data in the most suitable form for training.

%----------------------------------------------------------------


\section{Transformation of Numerical Features}\label{sec:transformation-of-numerical-features}

Upon analyzing the numerical variables during the exploratory phase, it was observed that several features, particularly Age, NCP (Number of Main Meals), FCVC (Frequency of Vegetable Consumption), and CH2O (Daily Water Intake), displayed right-skewed distributions.
Skewed data can adversely affect models, especially linear algorithms such as Logistic Regression, which assume a near-normal distribution of features.
Therefore, to improve normality and stabilize variance, various mathematical transformations were tested on these features, including logarithmic, square root, and Box–Cox transformations.

For the Age attribute, all three transformation methods were applied, and their effects on the distribution were analyzed using histograms and Q–Q plots.
The Box–Cox transformation produced the most symmetric and approximately normal distribution, significantly reducing skewness compared to log and square root transformations.
Therefore, the Box–Cox transformation was selected for Age as the optimal approach.
In contrast, for Weight, none of the applied transformations (log, square root, or Box– Cox) resulted in substantial improvement in distribution or model performance.
Since the weight feature already had a relatively stable and less skewed distribution, it was retained in its original form to preserve interpretability and avoid unnecessary distortion of values.
Mathematically, the Box–Cox transformation can be defined as:

\begin{equation}
    \centering
    x^\prime=\ \left\{
                   \begin{matrix}
                       \frac{x^\lambda}{\lambda}&if\ \lambda\neq0\\
                       \ln\funcapply(x)&if\ \lambda=0\\
                   \end{matrix}\right.
    \label{eq:eq_box_cox}
\end{equation}

where $x$ represents the feature value and 𝜆is a parameter determined through maximum likelihood estimation that minimizes skewness.
This transformation is particularly advantageous because it dynamically adapts to the data’s distribution rather than applying a fixed transformation across all variables.

%----------------------------------------------------------------


\section{Handling of Categorical Variables}\label{sec:handling-of-categorical-variables}

The dataset contained several categorical attributes such as Gender, family history of CVD, smoking habits, transportation means, and physical activity level, among others.
These features required encoding before being passed to machine learning models, as most algorithms cannot operate directly on non-numeric data.

Given the differences in how various algorithms interpret feature relationships, two distinct encoding strategies were employed-one for linear models and another for tree-based models:

\begin{description}
    \item[\textbf{For Linear Models:}]
    Linear models (including Regression-based, Distance-based and PRobability-based models) rely on distance-based calculations and assume linearity between predictors and target variables.
    To prevent the model from incorrectly inferring ordinal relationships among categories, One-Hot Encoding (OHE) was applied.
    In this encoding scheme, each category of a variable is converted into a separate binary feature, ensuring that no ordinal bias is introduced.
    Although OHE increases dimensionality, it allows the model to learn independent effects of each category accurately.
    After encoding, numerical features were standardized using StandardScaler to normalize feature magnitudes.

    \item[\textbf{For Tree-Based Models:}]
    Tree-based models are inherently insensitive to monotonic transformations and do not assume linear relationships between input features.
    Therefore, Label Encoding was employed for categorical variables in these models.
    Label Encoding assigns a unique integer to each category, which the model interprets as a discrete split criterion rather than a numerical ranking.
    This method is computationally efficient, reduces memory overhead, and avoids the curse of dimensionality introduced by OHE. This dual-encoding strategy ensured that each model type received appropriately formatted input features-linear models benefitted from interpretability and numerical uniformity through OHE, while tree-based models leveraged label-encoded data for computational efficiency and faster convergence
\end{description}

%----------------------------------------------------------------


\section{Feature Scaling}\label{sec:feature-scaling}

Feature scaling was another key aspect of the preprocessing pipeline.
Since numerical features were measured on different scales (e.g., Age in years, Height in meters, Weight in kilograms), unscaled data could bias models that depend on distance metrics or gradient optimization.
Hence, StandardScaler from scikit-learn was applied to all numerical features.
StandardScaler transforms each feature to have a mean of zero and a standard deviation of one, according to the formula:

\begin{equation}
    \centering
    x=\frac{x-\mu}{\sigma}
    \label{eq:eq_standard_Scaling}
\end{equation}

where $\mu$ and $\sigma$ represent the feature’s mean and standard deviation, respectively.
This step was especially crucial for algorithms such as Logistic Regression, K-Nearest Neighbors (KNN), and Naïve Bayes, which are sensitive to feature magnitude differences.
For tree-based models like Random Forest and XGBoost, scaling was not mandatory, as these models are invariant to monotonic transformations, but it was still maintained for consistency and comparative evaluation.

%----------------------------------------------------------------


\section{Derived and Interaction Features}\label{sec:derived-and-interaction-features}

In addition to transformation and encoding, several experiments were conducted to test the effect of derived features on model performance.
Derived features are new attributes constructed by combining or transforming existing ones to better represent underlying patterns.
For instance, features such as BMI-related interactions and agediet combinations were explored.
However, it was observed that introducing too many derived features led to marginal performance improvement but increased model complexity and risk of overfitting.
Therefore, only the most meaningful derived features-those contributing to consistent accuracy gains-were retained in the final dataset.

Comparative testing between datasets with and without derived features revealed that while certain linear models benefited slightly from derived interactions, ensemble models like XGBoost and LightGBM were already capable of learning non-linear interactions internally through feature splitting, making external feature engineering less impactful for them.

%----------------------------------------------------------------


\section{Preparation of Model-Specific Datasets}\label{sec:preparation-of-model-specific-datasets}

To maintain an efficient and organized workflow, two separate datasets were prepared post-feature engineering:

\begin{enumerate}
    \item \textbf{Dataset A (for Logistic Regression and other linear models):}
    \begin{itemize}
        \item Applied Box–Cox transformation to Age, NCP, FCVC, and CH2O.
        \item One-Hot Encoded all categorical features.
        \item Applied StandardScaler to all numerical features. \\ This dataset ensured numerical stability and feature independence for models relying on linear assumptions.
    \end{itemize}

    \item \textbf{Dataset B (for tree-based models such as XGBoost, LightGBM, Random Forest, and Decision Tree):}
    \begin{itemize}
        \item Applied Box–Cox transformation to the same set of skewed numerical features.
        \item Label Encoded categorical variables.
        \item Retained numerical features in their scaled form for consistency. \\ This dataset was designed to preserve the hierarchical and non-linear interpretive capability of tree-based algorithms while reducing unnecessary dimensionality.
    \end{itemize}
\end{enumerate}

%----------------------------------------------------------------


\section{Summary of Feature Engineering Outcomes}\label{sec:summary-of-feature-engineering-outcomes}

The feature engineering process significantly improved the quality and structure of the dataset, enabling better model learning and convergence.
Box–Cox transformation successfully normalized skewed distributions, enhancing the performance of models sensitive to feature variance.
The dual-encoding approach optimized categorical variable handling across different algorithm types, ensuring both interpretability and computational efficiency.
Additionally, careful evaluation of derived features helped maintain a balance between model complexity and generalization ability.

Through these systematic transformations, the dataset was effectively prepared for model training and validation, laying the foundation for robust and high-performing predictive models.
The improvements achieved during this stage were reflected in the final results, where optimized feature representation contributed to a remarkable increase in accuracy, culminating in the XGBoost model achieving 99\% accuracy on test data.

    \chapter{Model Training and Evaluation}\label{ch:model-training-and-evaluations}

Following the data preprocessing and feature engineering stages, the next phase involved training a series of machine learning models to predict the risk of cardiovascular disease (CVD) based on obesity-related attributes.
The primary objective was to build and evaluate multiple classifiers with varied underlying learning paradigms-ranging from simple linear models to advanced ensemble learners-to ensure a comprehensive understanding of model performance across different algorithmic families.


\section{Experimental Setup and Data Splitting}\label{sec:experimental-setup-and-data-splitting}
To ensure robust evaluation, the dataset was divided into training, validation, and testing subsets using a stratified split to maintain class balance across all partitions.

\begin{itemize}
    \item 70\% of the data was allocated for model development (training and validation), and
    \item 30\% was held out as a final test set for unbiased evaluation.
\end{itemize}

Additionally, Stratified K-Fold Cross-Validation (CV) (with k = 5) was used for more reliable performance estimation, particularly for hyperparameter tuning.
This approach helped mitigate issues of overfitting and provided a more stable measure of model generalization across different subsets of the data.

Given the moderate size of the dataset (approximately 15,000 samples, later expanded to 20,000 via SMOTE), the computational cost remained feasible, allowing multiple models to be trained and compared systematically.


\section{Model Selection Strategy}\label{sec:model-selection-strategy}
The project adopted a comparative modeling framework, where diverse algorithms were evaluated under a consistent preprocessing setup.
The models can broadly be classified into two categories:

\begin{enumerate}
    \item \textbf{Linear Models:}
    \begin{itemize}
        \item \textit{Logistic Regression} - served as a baseline, trained on one-hot encoded features and standardized numerical variables.
    \end{itemize}

    \item \textbf{Tree-based and Ensemble Models:}
    \begin{itemize}
        \item Decision Tree Classifier
        \item Random Forest Classifier
        \item AdaBoost Classifier
        \item XGBoost Classifier
        \item LightGBM Classifier
    \end{itemize}

    \item \textbf{Instance-Based and Probabilistic Models:}
    \begin{itemize}
        \item K-Nearest Neighbors (KNN)
        \item Naive Bayes Classifier
    \end{itemize}
\end{enumerate}

Each algorithm was chosen for its unique inductive bias and learning capability- logistic regression for its interpretability and linear separability, tree-based models for their ability to handle non-linear feature interactions, and boosting-based approaches for their capacity to reduce bias and variance through sequential learning.

\begin{itemize}
    \item \textbf{Logistic Regression}\\
    Logistic Regression is a classic model for binary classification problems — like predicting whether an email is spam or not.
    It estimates probabilities using the logistic (sigmoid) function, mapping predictions to a range between 0 and 1.
    Despite its name, it’s not a regression algorithm but a linear classifier.
    It’s simple, fast, and interpretable, but it struggles with complex, nonlinear data patterns.

    \importPlotFigure{figures/plot_Best LR Config.png}{Grid Search on Logistic Regression}{Best_LR_Config}
    \importPlotFigure{figures/plot_Confusion Matrix_ GSLR.png}{Confusion Matrix Logistic Regression}{ConfusionMatrix_GSLR}

    \item \textbf{Decision Tree Classifier}\\
    A Decision Tree splits data into branches based on feature values, forming a tree-like flowchart of decisions.
    It’s intuitive — easy to visualize and explain — and can handle both numeric and categorical data.
    However, single trees tend to overfit if not pruned, which is why ensemble versions like Random Forests or Boosting methods are often preferred for better generalization.

    \importPlotFigure{figures/plot_Best DT Config.png}{Grid Search on Decision Tree}{Best_DT_Config}
    \importPlotFigure{figures/plot_Confusion Matrix_ GSDT.png}{Confusion Matrix Decision Tree}{ConfusionMatrix_GSDT}

    \item \textbf{Random Forest Classifier}\\
    Random Forests combine many Decision Trees into one powerful model.
    Each tree is trained on a random sample of data and features, and the final prediction is made by majority vote.
    This reduces overfitting and improves accuracy.
    The trade-off?
    It’s slower and less interpretable than a single tree, but far more reliable in real-world tasks.

    \importPlotFigure{figures/plot_Best RF Config.png}{Grid Search on Random Forest}{Best_RF_Config}
    \importPlotFigure{figures/plot_Confusion Matrix_ GSRF.png}{Confusion Matrix Random Forest}{ConfusionMatrix_GSRF}

    \item \textbf{AdaBoost Classifier}\\
    AdaBoost (Adaptive Boosting) builds a series of weak models — often small decision trees — and focuses each new model on the mistakes of the previous ones.
    It combines them through weighted voting to form a strong overall predictor.
    AdaBoost can achieve high accuracy on clean data but tends to be sensitive to noise and outliers since it pays extra attention to difficult cases.

    \importPlotFigure{figures/plot_Best AB Config.png}{Grid Search on AdaBoost}{Best_AB_Config}
    \importPlotFigure{figures/plot_Confusion Matrix_ GSAB.png}{Confusion Matrix AdaBoost}{ConfusionMatrix_GSAB}

    \item \textbf{K-Nearest Neighbors (KNN)}\\
    KNN is a straightforward “lazy” learning algorithm — it doesn’t train a model upfront.
    Instead, when it needs to predict, it looks at the ‘k’ closest data points and classifies based on majority vote.
    It’s easy to understand and effective on smaller datasets, but it becomes slow and less accurate as data size grows, especially if the data isn’t well-scaled.

    \importPlotFigure{figures/plot_Best KNN Config.png}{Grid Search on K-Nearest Neighbors}{Best_KNN_Config}
    \importPlotFigure{figures/plot_Confusion Matrix_ GSKNN.png}{Confusion Matrix K-Nearest Neighbors}{ConfusionMatrix_GSKNN}

    \item \textbf{Naive Bayes}\\
    Naive Bayes is a simple probabilistic classifier based on Bayes’ Theorem, assuming all features are independent of each other.
    Despite that unrealistic assumption, it performs surprisingly well — especially in text classification tasks like spam filtering or sentiment analysis.
    It’s fast, efficient, and great for high-dimensional data, though it can miss subtle feature interactions.

    \importPlotFigure{figures/plot_Best GNB Config.png}{Grid Search on Naive Bayes}{Best_GNB_Config}
    \importPlotFigure{figures/plot_Confusion Matrix_ GSGNB.png}{Confusion Matrix Naive Bayes}{ConfusionMatrix_GSGNB}

    \item \textbf{XGBoost Classifier}\\
    XGBoost (Extreme Gradient Boosting) is a powerhouse in modern machine learning — known for its speed, accuracy, and efficiency.
    It builds trees sequentially, with each new tree correcting the errors of the previous ones using gradient-based optimization.
    With built-in regularization and smart handling of missing values, XGBoost consistently delivers top performance on structured datasets.

    \importPlotFigure{figures/plot_Best XGB Config.png}{Grid Search on XGBoost}{Best_XGB_Config}
    \importPlotFigure{figures/plot_Confusion Matrix_ GSXGB.png}{Confusion Matrix XGBoost}{ConfusionMatrix_GSXGB}

    \item \textbf{LightGBM Classifier}\\
    LightGBM, developed by Microsoft, is another gradient boosting framework designed for speed and scalability.
    Unlike XGBoost, it grows trees leaf-wise, which makes it faster and more memory-efficient — perfect for very large datasets.
    However, that same leaf-wise approach can sometimes cause overfitting on smaller data, so careful tuning is key.

    \importPlotFigure{figures/plot_Best LGB Config.png}{Grid Search on LightGBM}{Best_LGB_Config}
    \importPlotFigure{figures/plot_Confusion Matrix_ GSLGB.png}{Confusion Matrix LightGBM}{ConfusionMatrix_GSLGB}

\end{itemize}


\section{Hyperparameter Optimization}\label{sec:hyperparameter-optimization2}
Hyperparameter tuning played a critical role in enhancing model performance.

\begin{itemize}
    \item For Logistic Regression, Decision Tree, Random Forest, AdaBoost, KNN, and Naive Bayes, Grid Search Cross-Validation (GridSearchCV) from scikit-learn was emiloyed.
    The grid search spanned key parameters such as:
    \begin{itemize}
        \item Regularization strength (for logistic regression),
        \item Maximum depth and minimum samples split (for trees),
        \item Number of estimators and learning rate (for ensembles), and
        \item Optimal k value (for KNN).
    \end{itemize}

    \item For XGBoost and LightGBM, a more advanced optimization strategy was used - Optuna, an efficient hyperparameter optimization framework that applies Bayesian optimization with early stopping.
    This allowed exploration of larger hyperparameter spaces (e.g., learning rate, max\_depth, subsample ratio, colsample\_bytree, min\_child\_weight) without incurring exhaustive search costs.
    Both these models were trained using Stratified K-Fold CV to further strengthen generalization and avoid overfitting during the tuning process.
\end{itemize}


\section{Model Evaluation Metrics}\label{sec:model-evaluation-metrics}

Model performance was primarily evaluated using Accuracy, given the balanced nature of the dataset after SMOTE oversampling.
However, to ensure a holistic evaluation, Precision, Recall, and F1-Score were also monitored during crossvalidation to assess class-wise behavior and detect any residual imbalance bias.

The evaluation pipeline was designed to provide:

\begin{itemize}
    \item Cross-validation scores (mean ± standard deviation),
    \item Validation set performance, and
    \item Final test set performance after retraining the best configurations.
\end{itemize}

Visualization tools such as confusion matrices and ROC curves were used to interpret model predictions, especially to analyze false positive and false negative rates in a clinical context.


\section{Comparative Analysis and Observations}\label{sec:comparative-analysis-and-observations}
The comparative results highlighted clear trends across model families:

\begin{itemize}
    \item \textbf{Linear Model (Logistic Regression):}\\
    Despite feature standardization and transformation, logistic regression struggled to capture non-linear dependencies inherent in the dataset.
    Its interpretability was valuable, but its predictive power remained limited compared to ensemble models.

    \item \textbf{Tree-based Models (Decision Tree, Random Forest):}\\
    The decision tree classifier showed moderate accuracy but was prone to overfitting, while random forest performed significantly better due to aggregation of multiple trees, leading to improved generalization.

    \item \textbf{Boosting Models (AdaBoost, XGBoost, LightGBM):}\\
    These models outperformed all others, effectively handling both non-linear interactions and noisy data.
    Among them, XGBoost achieved the highest test accuracy of 93.2\%, outperforming all other models by a margin of approximately 3\%.
    LightGBM followed closely with marginally lower accuracy but higher training speed and efficiency.

    \item \textbf{Instance-based and Probabilistic Models (KNN, Naive Bayes):}\\
    KNN demonstrated reasonable accuracy but suffered from longer prediction times on larger data.
    Naive Bayes, although fast, performed suboptimally due to strong independence assumptions among features.
\end{itemize}


\section{Final Model Selection and Kaggle Submission}\label{sec:final-model-selection-and-kaggle-submission}
After detailed comparative evaluation, XGBoost was selected as the final model for test predictions and Kaggle submission due to its superior accuracy, robustness to noise, and balanced trade-off between bias and variance.

The model was retrained on the full training dataset (including validation data) using the best-found hyperparameters and then used to predict on the unseen Kaggle test set.
The final submission accuracy recorded on the leaderboard was consistent with the internal validation score, confirming the model’s reliability.


\section{Key Insights and Learnings}\label{sec:key-insights-and-learnings}

\begin{itemize}
    \item The inclusion of SMOTE-based oversampling improved model performance by ~3\%, particularly enhancing recall for minority classes.
    \item Removing outliers initially appeared to stabilize distributions but later reduced model generalization, reaffirming the importance of retaining certain noisy patterns for realistic data behavior.
    \item Feature transformations like Box-Cox significantly improved data normality and model interpretability for linear models.
    \item Ensemble-based learners, especially gradient boosting frameworks, proved highly effective for this dataset, indicating that non-linear relationships between lifestyle habits and cardiovascular risk factors were key predictive elements.
\end{itemize}


    \chapter{Results and Discussion}\label{ch:results}
The experimentation phase of this project yielded several meaningful insights into both the dataset characteristics and the relative strengths of different machine learning models when applied to obesity and cardiovascular disease (CVD) risk prediction.
Beyond raw accuracy figures, the results reveal important relationships between data preprocessing techniques, model complexity, and generalization performance.


\section{Quantitative Results}\label{sec:quantitative-results}

The comparative analysis of multiple models provided a clear performance hierarchy, highlighting the effectiveness of ensemble-based methods in capturing complex non-linear dependencies within the dataset.

\importTableFigure{tables/data_res_model_perf_comparison.csv}{Model Performance Compasrison}{data_res_model_perf_comparison}

(Values represent the mean accuracy across stratified 5-folds and final held-out test evaluations.)

The XGBoost classifier achieved the highest accuracy of 93.2\%, outperforming all other algorithms.
LightGBM followed closely with 92.6\%, while Random Forest remained competitive at 90.6\%.
These results underscore the strength of boosting-based methods in handling structured, moderately noisy tabular data.

\importPlotFigure{figures/plot_Model Accuracy Comparison.png}{Model Accuracy Comparison}{model_accuracy_comparison}


\section{Impact of Preprocessing and Feature Engineering}\label{sec:impact-of-preprocessing-and-feature-engineering}

One of the most significant findings of this project concerns the effect of data preprocessing and feature engineering on model performance.
The results demonstrate that even relatively minor transformations substantially influence downstream model behavior:

\textbf{Box-Cox Transformation:}\\
Among several tested transformations (logarithmic, square root, Box-Cox), the Box-Cox transformation yielded the most symmetric and approximately Gaussian distribution for features such as Age, NCP, FCVC, and CH2O.
This normalization improved the convergence and stability of linear models like logistic regression, leading to an increase of roughly 2–3\% in accuracy over the untransformed version.

\textbf{Outlier Handling:}\\
Experiments with outlier removal initially improved data distribution symmetry but negatively affected model generalization.
Given that the dataset itself was synthetically expanded from 2,000 to 15,000 samples (and further to 20,000 using SMOTE), a degree of inherent noise was expected.
Retaining these “noisy” samples allowed the model-particularly ensemble-based learners-to better capture variability representative of real-world populations.

\textbf{Oversampling using SMOTE:}\\
Synthetic Minority Oversampling Technique (SMOTE) was instrumental in balancing class distributions.
Oversampling from 15,000 to 20,000 samples enhanced the model’s robustness and improved the F1-score and recall metrics for underrepresented risk categories.
Notably, even the best-performing model, XGBoost, saw an approximate 3\% increase in accuracy after SMOTE was applied, demonstrating the utility of synthetic sample generation in health-related datasets where imbalance is common.

\textbf{Feature Encoding:}\\
Encoding choices were aligned with model architectures-One-Hot Encoding for linear models and Label Encoding for tree-based ones.
This separation ensured each model family received inputs in the most compatible numerical representation.
Tree-based learners like XGBoost and LightGBM handled categorical labels efficiently due to their split-based nature, while logistic regression benefited from the interpretability of one-hot encoded vectors.


\section{Model Behavior and Interpretations}\label{sec:model-behavior-and-interpretations}

The project findings highlight that model architecture directly influences how effectively the relationships between lifestyle, dietary, and behavioral attributes are captured.

\textbf{Linear Models (Logistic Regression):}\\
These models performed adequately but failed to capture the intricate non-linear dependencies between obesity indicators (such as BMI, caloric intake, and activity level) and CVD risk.
Despite their simplicity and interpretability, their predictive limits were evident.

\textbf{Tree-Based Models:}\\
Decision trees, while interpretable, were prone to overfitting and instability under small perturbations.
Random Forest mitigated this through ensembling, producing smoother decision boundaries and improved robustness.

\textbf{Boosting Algorithms (XGBoost, LightGBM, AdaBoost):}\\
These methods outperformed all others by sequentially correcting previous errors and learning higher-order interactions.
Their gradient-based optimization allowed efficient handling of mixed feature types, outliers, and mild noise.
XGBoost’s regularization mechanisms (lambda and alpha parameters) further reduced overfitting risk, making it the most generalizable model in this study.

\textbf{Instance-Based and Probabilistic Models:}\\
KNN and Naïve Bayes showed limited scalability and generalization.
KNN was sensitive to the choice of distance metric and value of k, while Naïve Bayes’s independence assumption proved unrealistic for this multivariate health dataset.


\section{Evaluation Metrics Beyond Accuracy}\label{sec:evaluation-metrics-beyond-accuracy}\

Although overall accuracy served as the primary metric, other measures provided additional insights:

\textbf{Precision and Recall:}\\
SMOTE notably improved recall for minority classes, reflecting better sensitivity toward high-risk individuals-a crucial factor in healthcare prediction models.

\textbf{F1-Score:}\\
Balancing precision and recall, F1-scores consistently improved after oversampling and feature normalization, showing that the models not only improved accuracy but also fairness in class prediction.

\textbf{ROC-AUC Curves:}\\
ROC curves for the top models (Random Forest, LightGBM, XGBoost) showed AUC values above 0.95, suggesting high discriminative power and reliable ranking of risk probabilities.

\importPlotFigure{figures/plot_Metric Comparison over Top Models.png}{Metric Comparison over Top Models}{metric_comparison}

These results collectively demonstrate that the predictive pipeline is not only statistically sound but also meaningful in a health-data context, where misclassification costs may be asymmetric.


\section{Discussion of Findings}\label{sec:discussion-of-findings}

The findings align well with theoretical expectations and practical experiences from prior literature on health-risk modeling:

\textbf{Importance of Non-linearity:}\\
Cardiovascular risk is influenced by complex interactions among behavioral, physiological, and demographic factors.
Linear models oversimplify these relationships, while tree-based ensembles efficiently capture such dependencies.

\textbf{Noise and Realism:}\\
Retaining outliers, despite appearing statistically undesirable, introduced beneficial noise that helped models generalize better.
This supports the hypothesis that mild irregularities in data can represent realistic patient-level variation rather than mere errors.

\textbf{Synthetic Data and Bias:}\\
Since the dataset originated from a smaller base sample of around 2,000 entries, followed by noise addition and synthetic expansion, it inherently contains correlated and non-independent observations.
The success of ensemble learners in this setting demonstrates their robustness to mild data imperfections and sampling bias.

\textbf{Computational Efficiency vs. Accuracy:}\\
While LightGBM offered faster training with near-identical accuracy to XGBoost, the latter provided marginally better calibration and consistency across folds, justifying its selection for final submission.


\section{Limitations}\label{sec:limitations}

Despite the encouraging results, several limitations were observed:

The dataset, though diverse, is synthetically extended and may not fully represent real-world population variance.

The project relied on accuracy-based evaluation; real clinical models may demand cost-sensitive or risk-weighted metrics.

Model interpretability for complex ensembles remains limited despite high predictive power, suggesting future work could integrate explainable AI (XAI) methods such as SHAP or LIME for transparency.

    % -----------------------
    % Conclusion
    % -----------------------
    {
        \chapter*{Conclusion}
        \addcontentsline{toc}{chapter}{Conclusion}
        This project successfully demonstrated the application of a structured machine learning pipeline for predicting cardiovascular disease (CVD) risk based on obesity-related parameters.
        Through systematic data exploration, feature engineering, and model evaluation, the study established a comprehensive framework that balances interpretability, performance, and robustness.

        The results highlight that ensemble-based methods, particularly XGBoost, deliver the best predictive performance, achieving an accuracy of 93.27\% on the held-out test dataset.
        This improvement was driven by careful preprocessing steps, including Box-Cox transformation for normalization, SMOTE-based oversampling for class balance, and appropriate encoding strategies for categorical variables.
        The experiments also revealed that retaining certain outliers improved model generalization, reflecting the realistic variability of health-related data.

        Comparative analyses confirmed that while linear models provided valuable interpretability, they lacked the flexibility to capture the complex non-linear interactions among features such as age, dietary habits, and activity levels.
        Tree-based and boosting algorithms, on the other hand, effectively modeled these relationships and maintained robustness against noise.

        In conclusion, this work demonstrates that thoughtfully engineered preprocessing combined with advanced ensemble learning can yield highly reliable health risk prediction models.
        The findings underscore the potential of machine learning in preventive healthcare analytics and provide a foundation for future extensions, such as integrating explainability frameworks (e.g., SHAP or LIME) and testing the model on real-world clinical datasets to enhance its applicability and trustworthiness
    }

    % -----------------------
    % References
    % -----------------------
    {
        \chapter*{References}
        \addcontentsline{toc}{chapter}{References}
        \printbibliography

        \vspace{1em}
        \section*{Web Resources}
        \addcontentsline{toc}{section}{Web Resources}

        \begin{itemize}
            \item \url{https://www.kaggle.com/competitions/playground-series-s4e2/code}
            \item \url{https://optuna.org/}
            \item \url{https://xgboost.readthedocs.io/en/stable/}
            \item \url{https://lightgbm.readthedocs.io/en/latest/}
            \item \url{https://scikit-learn.org/stable/}
            \item \url{https://imbalanced-learn.org/stable/references/generated/imblearn.over_sampling.SMOTE.html}
            \item \url{https://pandas.pydata.org/docs/}
            \item \url{https://numpy.org/doc/stable/}
            \item \url{https://matplotlib.org/stable/contents.html}
            \item \url{https://seaborn.pydata.org/}
            \item \url{https://docs.python.org/3/}
        \end{itemize}
    }

\end{document}
